\subsection {Histogram (Biểu đồ tần suất)}
\label{graph:hist}

Biểu đồ Histogram là một biểu đồ hình cột dùng để biểu diễn phân bố tần suất hoặc tần số của một tập dữ liệu định lượng. Histogram chia dữ liệu thành các khoảng (còn gọi là \textbf{bins}) và thể hiện số lượng phần tử (hoặc tần suất) rơi vào mỗi khoảng.

\textbf{Các thành phần chính của biểu đồ Histogram}

\begin{itemize}
    \item \textbf{Trục hoành (trục $x$)}: thể hiện các \textbf{khoảng giá trị} (bins) — được chia đều hoặc không đều tuỳ vào dữ liệu.
    \item \textbf{Trục tung (trục $y$)}: thể hiện \textbf{số lượng} (tần số) hoặc \textbf{tần suất tương đối} trong mỗi bin.
    \item \textbf{Cột (bars)}: chiều cao của mỗi cột thể hiện số phần tử thuộc bin đó.
\end{itemize}

\textbf{Cách xây dựng Histogram}

\begin{enumerate}
    \item Xác định giá trị nhỏ nhất và lớn nhất trong tập dữ liệu.
    \item Chia dữ liệu thành $k$ khoảng (bins) theo công thức như Sturges:
    \[
        k = \lceil \log_2 n + 1 \rceil
    \]
    trong đó $n$ là số quan sát.
    \item Tính độ rộng mỗi khoảng (bin width):
    \[
        \text{Bin width} = \frac{\text{Max} - \text{Min}}{k}
    \]
    \item Đếm số lượng phần tử rơi vào mỗi bin.
    \item Vẽ biểu đồ cột với chiều cao tương ứng.
\end{enumerate}

\textbf{Ví dụ minh họa}

Cho tập dữ liệu:

\[
x = \{5,\ 7,\ 8,\ 9,\ 10,\ 10,\ 11,\ 13,\ 13,\ 15,\ 16,\ 18,\ 20\}
\]

\noindent
Số phần tử: \(n = 13\)  
Tính số bin theo công thức Sturges:

\[
k = \lceil \log_2 13 + 1 \rceil = \lceil 3.7 + 1 \rceil = 5
\]

\noindent
Khoảng giá trị: \([5,\ 20]\), độ rộng bin:

\[
\text{Bin width} = \frac{20 - 5}{5} = 3
\]

\noindent
Các bin: \([5-8), [8-11), [11-14), [14-17), [17-20]\)

\noindent
Đếm số phần tử trong từng bin và vẽ biểu đồ cột.

\textbf{Ứng dụng của Histogram}

\begin{itemize}
    \item Phân tích dạng phân bố dữ liệu: chuẩn, lệch trái, lệch phải.
    \item Kiểm tra sự tồn tại của ngoại lệ hoặc cụm giá trị bất thường.
    \item Là bước đầu tiên trong phân tích thống kê mô tả và trực quan hóa dữ liệu.
\end{itemize}

\subsection {Box Plot (Biểu đồ hộp)} 
\label{graph:boxplot}

Biểu đồ hộp (Box plot), còn gọi là biểu đồ hộp-tia (box-and-whisker plot), là một công cụ trực quan hóa dữ liệu rất hiệu quả trong thống kê mô tả. Nó cho phép mô tả phân bố của tập dữ liệu theo các tóm tắt tứ phân vị, đồng thời phát hiện các giá trị ngoại lệ (outliers).

\textbf{Các thành phần của biểu đồ hộp}

Một biểu đồ hộp gồm các thành phần chính:

\begin{itemize}
    \item \textbf{Q1 (Tứ phân vị thứ nhất)} – 25\% dữ liệu nhỏ hơn hoặc bằng.
    \item \textbf{Q2 (Trung vị – Median)} – 50\% dữ liệu nhỏ hơn hoặc bằng.
    \item \textbf{Q3 (Tứ phân vị thứ ba)} – 75\% dữ liệu nhỏ hơn hoặc bằng.
    \item \textbf{IQR (Interquartile Range)} – \( Q_3 - Q_1 \), phạm vi của 50\% dữ liệu trung tâm.
    \item \textbf{Whiskers (tia)} – thể hiện phạm vi dữ liệu không bị xem là ngoại lệ:
    \[
    \text{Lower whisker} = Q_1 - 1.5 \cdot \text{IQR}
    \]
    \[
    \text{Upper whisker} = Q_3 + 1.5 \cdot \text{IQR}
    \]
    \item \textbf{Outliers (ngoại lệ)} – các điểm nằm ngoài khoảng trên.
\end{itemize}

\textbf{Ý nghĩa biểu đồ hộp}

\begin{itemize}
    \item Thể hiện trực quan độ phân tán, độ lệch và tính đối xứng của dữ liệu.
    \item So sánh phân bố giữa nhiều nhóm.
    \item Phát hiện nhanh các giá trị ngoại lệ.
\end{itemize}

\subsection*{Ví dụ minh họa}

Giả sử có tập dữ liệu:

\[
x = \{1,\ 3,\ 4,\ 5,\ 6,\ 7,\ 8,\ 10,\ 12,\ 15\}
\]

\noindent Các giá trị thống kê:

\begin{align*}
    Q_1 &= 4 \\
    Q_2 &= 6 \\
    Q_3 &= 10 \\
    \text{IQR} &= Q_3 - Q_1 = 6 \\
    \text{Whisker dưới} &= 4 - 1.5 \cdot 6 = -5 \\
    \text{Whisker trên} &= 10 + 1.5 \cdot 6 = 19
\end{align*}

\noindent Vì tất cả giá trị thuộc khoảng \([-5, 19]\), không có ngoại lệ trong tập dữ liệu.

\textbf{Ứng dụng}

\begin{itemize}
    \item So sánh sự phân bố dữ liệu giữa nhiều nhóm (ví dụ: điểm kiểm tra giữa các lớp).
    \item Phân tích nhanh độ lệch (skewness) và độ biến thiên.
    \item Dùng trong các biểu đồ thống kê y tế, tài chính, sản xuất và học máy.
\end{itemize}

\subsection {Scatter Plot (Biểu đồ phân tán)}
\label{graph:scatter}

Biểu đồ phân tán (Scatter Plot) là một biểu đồ dạng điểm, dùng để biểu diễn mối quan hệ giữa hai biến số định lượng. Mỗi điểm trên biểu đồ biểu thị một cặp giá trị \((x_i, y_i)\) của hai biến.

\textbf{Cấu trúc của biểu đồ}

\begin{itemize}
    \item \textbf{Trục hoành (Ox)}: biểu diễn biến độc lập (hoặc biến đầu vào) \(x\).
    \item \textbf{Trục tung (Oy)}: biểu diễn biến phụ thuộc (hoặc biến đầu ra) \(y\).
    \item \textbf{Các điểm dữ liệu}: mỗi điểm \((x_i, y_i)\) là một quan sát trong tập dữ liệu.
\end{itemize}

\textbf{Ý nghĩa}

\begin{itemize}
    \item Cho phép xác định trực quan mối quan hệ giữa hai biến:
    \begin{itemize}
        \item Quan hệ tuyến tính dương (cùng tăng).
        \item Quan hệ tuyến tính âm (một tăng, một giảm).
        \item Không có mối quan hệ rõ ràng.
    \end{itemize}
    \item Phát hiện xu hướng, cụm dữ liệu, điểm bất thường (outliers).
\end{itemize}

\textbf{Ví dụ minh họa}

Cho tập dữ liệu gồm chiều cao (cm) và cân nặng (kg) của 5 người:

\[
\begin{array}{|c|c|}
\hline
\text{Chiều cao (x)} & \text{Cân nặng (y)} \\
\hline
160 & 50 \\
165 & 55 \\
170 & 60 \\
175 & 66 \\
180 & 72 \\
\hline
\end{array}
\]

\noindent Khi vẽ các điểm \((x, y)\) này trên mặt phẳng tọa độ, ta có thể thấy một xu hướng tuyến tính dương — tức là người cao hơn thường nặng hơn.

\textbf{Ứng dụng}

\begin{itemize}
    \item Kiểm tra mối tương quan giữa hai biến trong phân tích thống kê và hồi quy.
    \item Trực quan hóa dữ liệu trong các lĩnh vực như kinh tế, sinh học, kỹ thuật.
    \item Dùng để phát hiện ngoại lệ hoặc cấu trúc đặc biệt trong dữ liệu.
\end{itemize}

\subsection{Heatmap (Biểu đồ nhiệt)}
\label{graph:heatmap}

Biểu đồ nhiệt (Heatmap) là một hình thức trực quan hóa dữ liệu trong đó các giá trị được biểu diễn thông qua màu sắc. Mỗi ô vuông trong biểu đồ tương ứng với một giá trị số trong bảng dữ liệu, với màu sắc càng đậm hoặc thay đổi sắc độ thể hiện giá trị càng lớn hoặc nhỏ.

\textbf{Thành phần chính của biểu đồ Heatmap}

\begin{itemize}
    \item \textbf{Trục hoành và trục tung}: đại diện cho hai chiều của ma trận dữ liệu (ví dụ: biến hoặc nhóm).
    \item \textbf{Mỗi ô (cell)}: biểu thị một giá trị số cụ thể được ánh xạ thành màu sắc.
    \item \textbf{Thang màu (color scale)}: chỉ thị giá trị nhỏ đến lớn bằng các gam màu từ nhạt đến đậm, lạnh đến nóng...
\end{itemize}

\textbf{Ứng dụng của Heatmap}

\begin{itemize}
    \item Phân tích tương quan giữa các biến (corr matrix).
    \item Biểu diễn cường độ hoạt động, tần suất, mật độ (density).
    \item Phân tích dữ liệu lớn có cấu trúc ma trận như trong học máy, sinh học, tài chính.
    \item So sánh giá trị giữa các nhóm, hạng mục trong một bảng nhiều chiều.
\end{itemize}

\textbf{Ví dụ minh họa}

Giả sử ta có bảng tương quan giữa các biến:

\[
\begin{bmatrix}
1.00 & 0.85 & 0.30 \\
0.85 & 1.00 & 0.60 \\
0.30 & 0.60 & 1.00 \\
\end{bmatrix}
\]

Biểu đồ nhiệt sẽ dùng các màu khác nhau để thể hiện độ tương quan (càng gần 1, màu càng đậm).


\subsection{Pie Plot (Biểu đồ tròn)}
\label{graph:pie}

Biểu đồ tròn (Pie Plot) là một dạng biểu đồ hình tròn dùng để biểu diễn tỷ lệ phần trăm (\%) của các thành phần trong một tổng thể. Mỗi phần hình quạt tương ứng với một hạng mục và có diện tích tỉ lệ với tỷ trọng của hạng mục đó.

\textbf{Đặc điểm chính}

\begin{itemize}
    \item Tổng các phần của biểu đồ luôn bằng \(100\%\) hoặc \(360^\circ\).
    \item Kích thước của mỗi phần được tính theo:
    \[
    \theta_i = \frac{x_i}{\sum x_i} \cdot 360^\circ
    \]
    Trong đó:
    \begin{itemize}
        \item \( x_i \): giá trị của nhóm thứ \(i\),
        \item \( \theta_i \): góc của nhóm thứ \(i\) trong biểu đồ.
    \end{itemize}
    \item Mỗi phần thường có màu sắc và nhãn riêng để dễ phân biệt.
\end{itemize}

\textbf{Ví dụ minh họa}

Xét tỉ lệ sinh viên theo ngành học trong một lớp:

\begin{itemize}
    \item Kỹ thuật: 40 sinh viên
    \item Kinh tế: 30 sinh viên
    \item Sư phạm: 20 sinh viên
    \item Luật: 10 sinh viên
\end{itemize}

Tổng số sinh viên: \(100\)

\begin{itemize}
    \item Kỹ thuật: \(40\%\) → \(144^\circ\)
    \item Kinh tế: \(30\%\) → \(108^\circ\)
    \item Sư phạm: \(20\%\) → \(72^\circ\)
    \item Luật: \(10\%\) → \(36^\circ\)
\end{itemize}

\textbf{Ứng dụng}

\begin{itemize}
    \item Hiển thị thành phần tỷ lệ của các nhóm dữ liệu rời rạc.
    \item Thường dùng trong báo cáo tài chính, thống kê khảo sát, thống kê dân số.
    \item Thích hợp với dữ liệu ít nhóm (dưới 6 nhóm để dễ đọc).
\end{itemize}

\subsection{Bar Plot (Biểu đồ cột)}
\label{graph:bar}


Biểu đồ cột (Bar Plot) là một trong những dạng biểu đồ thống kê phổ biến, dùng để thể hiện và so sánh số lượng, tần suất hoặc giá trị giữa các nhóm hạng mục rời rạc (danh mục).

\textbf{Thành phần của biểu đồ cột}

\begin{itemize}
    \item \textbf{Trục hoành (trục $x$)}: biểu diễn các hạng mục (categories), ví dụ: các nhóm tuổi, khoa, khu vực, năm...
    \item \textbf{Trục tung (trục $y$)}: biểu diễn số lượng hoặc giá trị tương ứng của từng hạng mục.
    \item \textbf{Cột (bars)}: chiều cao của mỗi cột tương ứng với giá trị của hạng mục đó.
\end{itemize}

\textbf{Công thức (nếu dùng dữ liệu đếm)}

\[
\text{Chiều cao cột} = \text{Tần suất hoặc giá trị tương ứng} = x_i
\]

\textbf{Ví dụ minh họa}

Giả sử thống kê số lượng sinh viên ở các khoa như sau:

\begin{itemize}
    \item CNTT: 120
    \item Kinh tế: 80
    \item Y: 60
    \item Môi trường: 40
\end{itemize}

\subsection*{Ứng dụng của biểu đồ cột}

\begin{itemize}
    \item So sánh số lượng hoặc giá trị giữa các nhóm rời rạc.
    \item Dùng trong báo cáo thống kê, trực quan hóa dữ liệu, khảo sát thị trường,...
    \item Kết hợp với màu sắc hoặc phân nhóm để thể hiện thêm chiều thông tin.
\end{itemize}

\subsection{Line Plot (Biểu đồ đường)}
\label{graph:line}

Biểu đồ đường (Line Plot hoặc Line Chart) là một loại biểu đồ dùng để hiển thị dữ liệu dạng chuỗi (theo thời gian hoặc thứ tự) bằng cách nối các điểm dữ liệu bằng các đoạn thẳng. Đây là công cụ hiệu quả để biểu diễn xu hướng thay đổi của dữ liệu theo thời gian hoặc các giai đoạn.

\textbf{Thành phần của biểu đồ}

\begin{itemize}
    \item \textbf{Trục hoành (Ox)}: biểu diễn chuỗi thời gian hoặc thứ tự quan sát (ngày, tháng, năm, phiên, v.v.).
    \item \textbf{Trục tung (Oy)}: biểu diễn giá trị tương ứng tại từng thời điểm.
    \item \textbf{Các điểm dữ liệu (data points)}: là các cặp \((x_i, y_i)\).
    \item \textbf{Đường nối (lines)}: nối liên tiếp các điểm dữ liệu để thể hiện xu hướng.
\end{itemize}

\textbf{Ứng dụng}

\begin{itemize}
    \item Phân tích xu hướng theo thời gian (biến động giá, dân số, nhiệt độ, doanh thu...).
    \item So sánh sự thay đổi giữa nhiều biến theo thời gian.
    \item Trực quan hóa dữ liệu chuỗi thời gian trong thống kê, tài chính, khoa học, và học máy.
\end{itemize}

\subsection*{Ví dụ minh họa}

Xét số lượng khách truy cập website theo tháng:

\begin{center}
\begin{tabular}{|c|c|}
\hline
\textbf{Tháng} & \textbf{Số lượt truy cập (nghìn)} \\
\hline
1 & 15 \\
2 & 18 \\
3 & 21 \\
4 & 25 \\
5 & 24 \\
6 & 27 \\
\hline
\end{tabular}
\end{center}

Biểu đồ đường sẽ nối các điểm tương ứng: \((1, 15), (2, 18), \dots, (6, 27)\)
