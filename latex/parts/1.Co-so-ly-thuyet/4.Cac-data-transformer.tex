\subsection {Chuẩn hóa dữ liệu với Min-Max Scaler}
\label{scaler:minmax}

Trong học máy, việc chuẩn hóa dữ liệu đầu vào giúp mô hình học hiệu quả hơn, đặc biệt với các mô hình nhạy cảm với thang đo (như KNN, SVM, hồi quy Logistic, mạng nơ-ron,...).

Một trong các phương pháp chuẩn hóa phổ biến là \textbf{Min-Max Scaling}, hay còn gọi là \textit{feature normalization}.

\textbf{Công thức Min-Max Scaling}

\[
x_i^{\text{scaled}} = \frac{x_i - x_{\min}}{x_{\max} - x_{\min}}
\]

Trong đó:
\begin{itemize}
    \item \( x_i \): giá trị gốc của đặc trưng.
    \item \( x_{\min} \): giá trị nhỏ nhất của đặc trưng trong tập dữ liệu.
    \item \( x_{\max} \): giá trị lớn nhất của đặc trưng.
    \item \( x_i^{\text{scaled}} \): giá trị sau khi chuẩn hóa.
\end{itemize}

\textbf{Ý nghĩa}

Min-Max Scaler biến đổi các đặc trưng về khoảng giá trị \([0, 1]\), giúp:

\begin{itemize}
    \item Tăng tốc độ hội tụ của thuật toán tối ưu.
    \item Tránh hiện tượng một vài đặc trưng chi phối quá trình học do thang đo lớn.
    \item Dữ liệu có cùng thang đo, phù hợp với các mô hình tính toán khoảng cách.
\end{itemize}

\textbf{Ví dụ}

Cho tập dữ liệu: \( x = [20, 40, 60, 80, 100] \)

\begin{itemize}
    \item \( x_{\min} = 20 \), \( x_{\max} = 100 \)
    \item \( x_3 = 60 \Rightarrow x_3^{\text{scaled}} = \frac{60 - 20}{100 - 20} = \frac{40}{80} = 0.5 \)
\end{itemize}

\textbf{Lưu ý}

\begin{itemize}
    \item Cần áp dụng giá trị \( x_{\min}, x_{\max} \) từ tập huấn luyện cho cả tập kiểm tra.
    \item Không nên dùng Min-Max Scaling nếu dữ liệu chứa nhiều ngoại lệ (nên dùng Robust Scaler).
\end{itemize}

\subsection{Chuẩn hóa dữ liệu với Standard Scaler}
\label{scaler:standard}

Trong học máy, việc chuẩn hóa dữ liệu giúp các mô hình hoạt động hiệu quả hơn, đặc biệt với các thuật toán nhạy cảm với độ lớn của đặc trưng như: hồi quy Logistic, KNN, SVM, mạng nơ-ron,...

Một kỹ thuật chuẩn hóa phổ biến là \textbf{Standard Scaler} – biến đổi dữ liệu sao cho có trung bình bằng 0 và độ lệch chuẩn bằng 1.

\textbf{Công thức chuẩn hóa Standard Scaler}

\[
x_i^{\text{scaled}} = \frac{x_i - \mu}{\sigma}
\]

Trong đó:
\begin{itemize}
    \item \( x_i \): giá trị gốc của đặc trưng.
    \item \( \mu \): trung bình (mean) của đặc trưng.
    \item \( \sigma \): độ lệch chuẩn (standard deviation).
    \item \( x_i^{\text{scaled}} \): giá trị sau khi chuẩn hóa.
\end{itemize}

\textbf{Ý nghĩa}

\begin{itemize}
    \item Dữ liệu sau chuẩn hóa có phân phối chuẩn hóa chuẩn: \( \mathcal{N}(0, 1) \).
    \item Trung bình bằng 0 và phương sai bằng 1 giúp mô hình học ổn định hơn.
\end{itemize}

\textbf{Ví dụ}

Cho tập dữ liệu: \( x = [10, 12, 14, 16, 18] \)

\begin{itemize}
    \item Trung bình \( \mu = 14 \), độ lệch chuẩn \( \sigma = \sqrt{\frac{(4^2 + 2^2 + 0^2 + 2^2 + 4^2)}{5}} = \sqrt{8} \approx 2.828 \)
    \item \( x_1^{\text{scaled}} = \frac{10 - 14}{\sigma} \approx \frac{-4}{2.828} \approx -1.414 \)
\end{itemize}

\textbf{Lưu ý}

\begin{itemize}
    \item Nên tính \( \mu \) và \( \sigma \) trên tập huấn luyện, rồi áp dụng lên cả tập kiểm tra.
    \item Phù hợp nếu dữ liệu có phân phối gần chuẩn.
    \item Không phù hợp khi có nhiều ngoại lệ → khi đó dùng RobustScaler.
\end{itemize}

\subsection{Chuẩn hóa dữ liệu với Robust Scaler}
\label{scaler:robust}

\textbf{Robust Scaler} là một kỹ thuật chuẩn hóa dữ liệu giúp giảm ảnh hưởng của các giá trị ngoại lệ (outliers). Khác với Min-Max Scaler (dựa trên max/min) hay Standard Scaler (dựa trên trung bình/độ lệch chuẩn), Robust Scaler sử dụng các số liệu thống kê mạnh như \textbf{trung vị} (median) và \textbf{IQR – khoảng tứ phân vị}.

\textbf{Công thức chuẩn hóa}

\[
x_i^{\text{scaled}} = \frac{x_i - \text{Median}(x)}{\text{IQR}(x)}
\]

Trong đó:
\begin{itemize}
    \item \( \text{Median}(x) \): trung vị của đặc trưng.
    \item \( \text{IQR}(x) = Q_3 - Q_1 \): khoảng tứ phân vị (Interquartile Range), với:
        \begin{itemize}
            \item \( Q_1 \): phân vị thứ 25\% (lower quartile)
            \item \( Q_3 \): phân vị thứ 75\% (upper quartile)
        \end{itemize}
    \item \( x_i^{\text{scaled}} \): giá trị sau chuẩn hóa.
\end{itemize}

\textbf{Ưu điểm}

\begin{itemize}
    \item Không bị ảnh hưởng mạnh bởi ngoại lệ.
    \item Giữ nguyên phân phối cơ bản của dữ liệu phần lớn.
    \item Phù hợp với các mô hình yêu cầu dữ liệu chuẩn hóa như KNN, SVM, Logistic Regression.
\end{itemize}

\textbf{Ví dụ}

Giả sử đặc trưng \( x = [10, 12, 14, 16, 100] \)

\begin{itemize}
    \item \( \text{Median}(x) = 14 \)
    \item \( Q_1 = 12, Q_3 = 16 \Rightarrow \text{IQR} = 4 \)
    \item \( x_5^{\text{scaled}} = \frac{100 - 14}{4} = 21.5 \)
    \item Trong khi Min-Max sẽ nén dữ liệu về [0,1], nhưng sẽ bị kéo lệch vì 100 là ngoại lệ.
\end{itemize}

\textbf{Lưu ý}

\begin{itemize}
    \item Nên tính Median và IQR từ tập huấn luyện và áp dụng cho tập kiểm tra.
    \item Không thích hợp nếu dữ liệu đã phân bố chuẩn và không có ngoại lệ → dùng StandardScaler.
\end{itemize}

\subsection{Chuẩn hóa dữ liệu với MaxAbsScaler}
\label{scaler:maxabs}

\textbf{MaxAbsScaler} là một kỹ thuật chuẩn hóa dữ liệu theo phương pháp co dãn tuyến tính (linear scaling), sử dụng giá trị tuyệt đối lớn nhất trong từng đặc trưng để đưa dữ liệu về khoảng \([-1, 1]\), mà vẫn giữ nguyên dấu gốc của dữ liệu.

\textbf{Công thức chuẩn hóa}

\[
x_i^{\text{scaled}} = \frac{x_i}{\max \left( |x| \right)}
\]

Trong đó:
\begin{itemize}
    \item \( x_i \): giá trị ban đầu của đặc trưng.
    \item \( \max(|x|) \): giá trị tuyệt đối lớn nhất trong đặc trưng đó.
    \item \( x_i^{\text{scaled}} \): giá trị sau chuẩn hóa, nằm trong khoảng \( [-1, 1] \).
\end{itemize}

\textbf{Đặc điểm}

\begin{itemize}
    \item Dữ liệu được chia cho giá trị tuyệt đối lớn nhất, nên không làm thay đổi sparsity (độ thưa).
    \item Giữ nguyên dấu ban đầu của dữ liệu.
    \item Thích hợp với dữ liệu có giá trị cả âm và dương, đặc biệt là dữ liệu sparse (rất nhiều giá trị 0).
\end{itemize}

\textbf{Ví dụ}

Cho đặc trưng \( x = [-3, -1, 0, 2, 4] \):

\[
\max(|x|) = 4, \quad x^{\text{scaled}} = \left[ \frac{-3}{4}, \frac{-1}{4}, 0, \frac{2}{4}, \frac{4}{4} \right] = [-0.75, -0.25, 0, 0.5, 1.0]
\]

\textbf{Lưu ý}

\begin{itemize}
    \item Phù hợp với dữ liệu sparse (tránh tạo thêm giá trị khác 0).
    \item Không xử lý outlier – giá trị cực trị vẫn ảnh hưởng đến việc scale.
    \item Không trung tâm hóa dữ liệu (không đưa trung bình về 0).
\end{itemize}

\subsection{Chuẩn hóa dữ liệu với Quantile Transformer}
\label{scaler:quantile}

\textbf{Quantile Transformer} là một phương pháp chuẩn hóa dữ liệu phi tuyến, chuyển đổi phân phối gốc của dữ liệu về một phân phối mục tiêu (thường là phân phối chuẩn hoặc phân phối đều) bằng cách sử dụng \textbf{phân vị (quantiles)}.

\textbf{Ý tưởng chính}

Quantile Transformer thực hiện hai bước:

\begin{enumerate}
    \item Ánh xạ mỗi giá trị dữ liệu sang giá trị phân vị tương ứng trong tập huấn luyện.
    \item Biến đổi các phân vị đó sang giá trị trong phân phối mục tiêu: \textit{uniform} hoặc \textit{normal}.
\end{enumerate}

\textbf{Biến đổi phân vị}

Giả sử dữ liệu ban đầu là \( x \in \mathbb{R} \), ta có:

\[
x_i^{\text{scaled}} = F_{\text{target}}^{-1}(F_{\text{empirical}}(x_i))
\]

Trong đó:
\begin{itemize}
    \item \( F_{\text{empirical}}(x_i) \): phân phối tích lũy thực nghiệm (ECDF) – giá trị phân vị của \( x_i \)
    \item \( F_{\text{target}}^{-1} \): hàm nghịch đảo của phân phối mục tiêu (chuẩn hoặc đều)
    \item \( x_i^{\text{scaled}} \): giá trị sau khi chuẩn hóa
\end{itemize}

\textbf{Phân phối đầu ra}

\begin{itemize}
    \item \textbf{Uniform} (phân phối đều): đưa dữ liệu về khoảng \([0, 1]\)
    \item \textbf{Normal} (phân phối chuẩn): đưa dữ liệu về phân phối chuẩn \( \mathcal{N}(0, 1) \)
\end{itemize}

\textbf{Ưu điểm}

\begin{itemize}
    \item Biến đổi phân phối lệch về phân phối chuẩn/đều → cải thiện hiệu suất mô hình.
    \item Giảm tác động của ngoại lệ (outliers).
    \item Phù hợp với các mô hình tuyến tính hoặc giả định phân phối chuẩn.
\end{itemize}

\textbf{Nhược điểm}

\begin{itemize}
    \item Là phép biến đổi phi tuyến → có thể làm mất quan hệ tuyến tính gốc.
    \item Dễ bị overfit nếu tập huấn luyện nhỏ.
\end{itemize}

\textbf{Ví dụ đơn giản}

Cho dữ liệu \( x = [10, 100, 1000] \)

\begin{itemize}
    \item \( 10 \rightarrow \text{quantile} = 0.0 \)
    \item \( 100 \rightarrow \text{quantile} = 0.5 \)
    \item \( 1000 \rightarrow \text{quantile} = 1.0 \)
    \item Nếu dùng phân phối chuẩn: giá trị tương ứng sẽ là \( [-\infty, 0, +\infty] \)
\end{itemize}
