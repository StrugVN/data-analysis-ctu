\subsection {Accuracy (Độ chính xác)}
\label{eval:acc}
Accuracy là độ đo đơn giản nhất dùng để đánh giá hiệu suất của mô hình phân loại, được tính bằng tỷ lệ giữa số lượng dự đoán đúng và tổng số dự đoán:

\begin{equation}
\text{Accuracy} = \frac{TP + TN}{TP + TN + FP + FN}
\end{equation}

Ví dụ với bài toán phân loại \textit{Cat/Non-cat}, giả sử có 90 ảnh \textit{Cat} được phân loại đúng, 10 ảnh \textit{Cat} bị phân loại sai, 940 ảnh \textit{Non-cat} phân loại đúng và 60 ảnh \textit{Non-cat} bị phân loại sai. Khi đó:

\begin{equation}
\text{Accuracy} = \frac{90 + 940}{1000 + 100} = 93.6\%
\end{equation}

Tuy nhiên, Accuracy không phản ánh cụ thể mô hình xử lý tốt nhóm nào, dễ gây hiểu nhầm nếu dữ liệu mất cân bằng.

\subsection {F1-score}
\label{eval:f1}
Khi cả Precision và Recall đều quan trọng, ta sử dụng F1-score, là trung bình điều hòa của Precision và Recall:

\begin{equation}
\text{F1-score} = 2 \cdot \frac{\text{Precision} \cdot \text{Recall}}{\text{Precision} + \text{Recall}}
\end{equation}

F1-score giúp cân bằng giữa Precision và Recall, đặc biệt hữu ích khi dữ liệu bị mất cân bằng.

\subsection {Precision}
\label{eval:prec}
Precision đo lường tỷ lệ dự đoán dương tính đúng trong tất cả các dự đoán dương tính:

\begin{equation}
\text{Precision} = \frac{TP}{TP + FP}
\end{equation}

Áp dụng cho bài toán \textit{Cat/Non-cat}:

\begin{align*}
\text{Precision}_{\text{Cat}} &= \frac{90}{90 + 60} = 60\% \\
\text{Precision}_{\text{Non-cat}} &= \frac{940}{940 + 10} = 98.9\%
\end{align*}

Precision giúp đánh giá mô hình có dự đoán nhầm quá nhiều hay không trong nhóm dương tính.

\subsection {Recall}
\label{eval:prec}
Recall (Sensitivity, True Positive Rate) là tỷ lệ phát hiện đúng các mẫu dương tính thực sự:

\begin{equation}
\text{Recall} = \frac{TP}{TP + FN}
\end{equation}

Ví dụ:

\begin{align*}
\text{Recall}_{\text{Cat}} &= \frac{90}{90 + 10} = 90\% \\
\text{Recall}_{\text{Non-cat}} &= \frac{940}{940 + 60} = 94\%
\end{align*}

Recall cao thể hiện mô hình ít bỏ sót mẫu dương tính thực sự.

\subsection {ROC AUC}
\label{eval:rocauc}
\textbf{ROC Curve} (Receiver Operating Characteristic Curve) là biểu đồ thể hiện mối quan hệ giữa TPR và FPR:

\begin{align}
\text{TPR} &= \frac{TP}{TP + FN} \quad (\text{Recall}) \\
\text{FPR} &= \frac{FP}{FP + TN}
\end{align}

\textbf{AUC} (Area Under the Curve) là diện tích dưới đường cong ROC, biểu thị tổng quát khả năng phân loại của mô hình. AUC có giá trị từ 0 đến 1:
\begin{itemize}
    \item AUC gần 1: mô hình phân loại tốt.
    \item AUC gần 0.5: mô hình phân loại ngẫu nhiên.
\end{itemize}

Ví dụ: Với đầu ra xác suất $[0.45, 0.6, 0.7, 0.3]$, ta thay đổi ngưỡng để xác định nhãn phân loại. Khi vẽ TPR và FPR ứng với nhiều ngưỡng, ta có thể vẽ được ROC Curve. Diện tích dưới đường cong chính là AUC.

\textbf{Giải thích chi tiết các tham số}

Các độ đo như Accuracy, Precision, Recall, F1-score,... đều dựa trên \textbf{Ma trận nhầm lẫn (Confusion Matrix)}. Ma trận này được xây dựng dựa trên so sánh giữa nhãn dự đoán của mô hình và nhãn thực tế. Đối với bài toán phân loại nhị phân (binary classification), ma trận nhầm lẫn có 4 phần tử chính:

\begin{center}
\begin{tabular}{|c|c|c|}
\hline
 & \textbf{Dự đoán: Positive} & \textbf{Dự đoán: Negative} \\
\hline
\textbf{Thực tế: Positive} & TP (True Positive) & FN (False Negative) \\
\hline
\textbf{Thực tế: Negative} & FP (False Positive) & TN (True Negative) \\
\hline
\end{tabular}
\end{center}

\begin{itemize}
    \item \textbf{TP (True Positive)}: Số mẫu dương tính được mô hình dự đoán đúng là dương tính.
    \item \textbf{TN (True Negative)}: Số mẫu âm tính được mô hình dự đoán đúng là âm tính.
    \item \textbf{FP (False Positive)}: Số mẫu âm tính bị mô hình dự đoán nhầm là dương tính (hay còn gọi là lỗi loại I).
    \item \textbf{FN (False Negative)}: Số mẫu dương tính bị mô hình dự đoán nhầm là âm tính (hay còn gọi là lỗi loại II).
\end{itemize}


\subsection {Mean Squared Error (MAE)}
\label{eval:mse}

MSE (Mean Squared Error) là một trong những độ đo phổ biến nhất dùng để đánh giá hiệu suất của các mô hình hồi quy. Nó đo lường sai số trung bình bình phương giữa giá trị dự đoán và giá trị thực tế.

Giả sử với một bộ dữ liệu gồm $n$ mẫu, giá trị thực tế được ký hiệu là $y_i$, giá trị dự đoán là $\hat{y}_i$, thì MSE được tính theo công thức:

\begin{equation}
\text{MSE} = \frac{1}{n} \sum_{i=1}^{n} (y_i - \hat{y}_i)^2
\end{equation}

MSE phạt nặng các sai số lớn do sai số được bình phương. Do đó, MSE nhạy cảm với các điểm ngoại lai (\textit{outliers}).

\vspace{0.3cm}
\noindent\textbf{Ví dụ:} Trong bài toán dự đoán giá nhà, $y_i$ là giá trị thực tế của căn nhà thứ $i$, và $\hat{y}_i$ là giá trị dự đoán. MSE sẽ đo lường khoảng cách trung bình bình phương giữa các giá trị này.


\subsection {Mean Absolute Error (MAE)}
\label{eval:mae}
MAE (Mean Absolute Error) là một độ đo khác, đánh giá sai số trung bình tuyệt đối giữa giá trị thực tế và giá trị dự đoán:

\begin{equation}
\text{MAE} = \frac{1}{n} \sum_{i=1}^{n} |y_i - \hat{y}_i|
\end{equation}

Không giống như MSE, MAE sử dụng trị tuyệt đối thay vì bình phương, do đó nó ít bị ảnh hưởng bởi các điểm ngoại lai. MAE được coi là một metric \textit{mạnh hơn} khi mô hình cần khả năng chống chịu với dữ liệu nhiễu.

\vspace{0.3cm}
\noindent\textbf{So sánh MSE và MAE:}
\begin{itemize}
    \item \textbf{MSE} nhấn mạnh các lỗi lớn (do bình phương sai số), phù hợp khi muốn phạt mạnh các dự đoán sai nhiều.
    \item \textbf{MAE} phản ánh mức sai lệch trung bình thực sự, bền vững hơn với các ngoại lai.
\end{itemize}

\subsection {Root Mean Squared Error (RMSE)}
\label{eval:rmse}
RMSE (Root Mean Squared Error) là một metric đánh giá độ lệch giữa giá trị dự đoán và giá trị thực tế trong các bài toán hồi quy. RMSE chính là căn bậc hai của MSE (Mean Squared Error), do đó vẫn giữ đặc tính "phạt nặng" các sai số lớn, nhưng giá trị của nó có cùng đơn vị với biến đầu ra.

\textbf{Công thức tính RMSE}

Giả sử có $n$ mẫu dữ liệu, $y_i$ là giá trị thực tế, và $\hat{y}_i$ là giá trị dự đoán của mô hình. Khi đó, RMSE được tính như sau:

\begin{equation}
\text{RMSE} = \sqrt{ \frac{1}{n} \sum_{i=1}^{n} (y_i - \hat{y}_i)^2 }
\end{equation}

\textbf{Ý nghĩa}

\begin{itemize}
    \item RMSE biểu thị độ lệch chuẩn của sai số dự đoán.
    \item Đơn vị của RMSE giống với đơn vị của biến đầu ra, giúp dễ diễn giải trong thực tế.
    \item RMSE nhạy cảm với các điểm ngoại lai (outliers) hơn so với MAE, do bình phương sai số.
\end{itemize}

\subsubsection*{So sánh RMSE và MSE}

\begin{itemize}
    \item \textbf{MSE}: thường được dùng cho mục đích tối ưu trong huấn luyện mô hình (ví dụ trong đạo hàm của hàm mất mát).
    \item \textbf{RMSE}: thường được dùng để báo cáo kết quả mô hình vì đơn vị dễ hiểu hơn.
\end{itemize}

\textbf{Ví dụ}

Nếu mô hình dự đoán giá nhà, trong đó giá trị thực tế và dự đoán tính bằng triệu đồng, thì RMSE = 50 nghĩa là sai số trung bình của mô hình là khoảng \textbf{50 triệu đồng}.


\subsection {Mean Absolute Percentage Error (MAPE)}
\label{eval:mape}
MAPE (Mean Absolute Percentage Error) là một độ đo thường được sử dụng trong các bài toán hồi quy để đánh giá sai số trung bình phần trăm giữa giá trị dự đoán và giá trị thực tế.

\textbf{Công thức tính}

Giả sử ta có $n$ mẫu dữ liệu, với $y_i$ là giá trị thực tế và $\hat{y}_i$ là giá trị dự đoán tương ứng. Công thức tính MAPE như sau:

\begin{equation}
\text{MAPE} = \frac{100\%}{n} \sum_{i=1}^{n} \left| \frac{y_i - \hat{y}_i}{y_i} \right|
\end{equation}

\textbf{Ý nghĩa}

\begin{itemize}
    \item MAPE biểu thị sai số trung bình theo phần trăm so với giá trị thực tế.
    \item MAPE dễ diễn giải và thường được sử dụng trong các bài toán dự báo thời gian (\textit{forecasting}).
    \item Ví dụ: MAPE = 8.5\% nghĩa là trung bình mô hình dự đoán sai lệch 8.5\% so với giá trị thực tế.
\end{itemize}

\textbf{Ưu và nhược điểm}

\begin{itemize}
    \item \textbf{Ưu điểm:}
        \begin{itemize}
            \item Diễn giải trực quan dưới dạng phần trăm.
            \item Không phụ thuộc vào đơn vị đo.
        \end{itemize}
    \item \textbf{Nhược điểm:}
        \begin{itemize}
            \item Không xác định được nếu có $y_i = 0$ (chia cho 0).
            \item Nhạy cảm với các giá trị thực nhỏ, dễ làm tăng sai số phần trăm.
        \end{itemize}
\end{itemize}

\textbf{Lưu ý khi sử dụng}

MAPE không phù hợp nếu tập dữ liệu chứa các giá trị thực tế bằng 0 hoặc gần 0. Trong các trường hợp này, nên dùng SMAPE (Symmetric MAPE) hoặc các metric khác như MAE, RMSE.

\subsection{Adjusted Rand Index (ARI)}
\textbf{Khái niệm}

Adjusted Rand Index (ARI) là một độ đo thống kê dùng để đánh giá mức độ tương đồng giữa hai tập phân cụm: một là kết quả phân cụm của mô hình, và một là nhãn thực (ground truth), nếu có. ARI điều chỉnh theo sự ngẫu nhiên, giúp khắc phục nhược điểm của chỉ số Rand gốc (Rand Index - RI), vốn có thể cho giá trị cao ngay cả khi hai phân cụm gần như ngẫu nhiên.

Chỉ số ARI được sử dụng phổ biến trong các bài toán phân cụm khi muốn so sánh mức độ chính xác của mô hình phân cụm không giám sát với phân nhóm thực tế.

\textbf{Công thức tính ARI}

Giả sử ta có:
\begin{itemize}
    \item $n$: Tổng số điểm dữ liệu.
    \item $X = \{X_1, X_2, \ldots, X_r\}$: Tập phân cụm thực tế gồm $r$ cụm.
    \item $Y = \{Y_1, Y_2, \ldots, Y_s\}$: Tập phân cụm dự đoán gồm $s$ cụm.
\end{itemize}

Gọi $n_{ij}$ là số phần tử nằm đồng thời trong cụm $X_i$ và $Y_j$.

Gọi:
\begin{align*}
a_i &= \sum_{j} n_{ij} \quad \text{(tổng số điểm trong cụm thực tế $X_i$)} \\
b_j &= \sum_{i} n_{ij} \quad \text{(tổng số điểm trong cụm dự đoán $Y_j$)}
\end{align*}

Khi đó, công thức ARI được tính như sau:

\begin{equation}
\text{ARI} = \frac{
    \sum_{ij} \binom{n_{ij}}{2}
    - \left[ \sum_i \binom{a_i}{2} \sum_j \binom{b_j}{2} \middle/ \binom{n}{2} \right]
}{
    \frac{1}{2} \left[ \sum_i \binom{a_i}{2} + \sum_j \binom{b_j}{2} \right]
    - \left[ \sum_i \binom{a_i}{2} \sum_j \binom{b_j}{2} \middle/ \binom{n}{2} \right]
}
\end{equation}

\textbf{Giải thích ý nghĩa}

\begin{itemize}
    \item ARI có giá trị nằm trong khoảng $[-1, 1]$.
    \item ARI = 1 khi hai phân cụm hoàn toàn khớp nhau.
    \item ARI $\approx$ 0 khi phân cụm gần như ngẫu nhiên.
    \item ARI < 0 có thể xảy ra nếu phân cụm kém hơn cả ngẫu nhiên.
\end{itemize}

\textbf{Ví dụ sử dụng}

Chẳng hạn trong bài toán phân cụm ảnh khuôn mặt hoặc nhóm khách hàng, ARI cho phép đánh giá mô hình phân cụm khi có nhãn thật để so sánh. Đây là công cụ hữu ích khi so sánh các thuật toán phân cụm như K-Means, DBSCAN, Agglomerative Clustering, v.v.
