\chapter*{Lời cám ơn}
\startcontents[chapters]
Trong suốt hơn hai tháng thực hiện nghiên cứu, nhóm chúng em đã nhận được sự quan tâm, giúp đỡ tận tình từ quý thầy cô và bạn bè. Nhờ sự chia sẻ kinh nghiệm, truyền đạt kiến thức quý báu ấy, nhóm đã có được nền tảng vững chắc để hoàn thành báo cáo.

Đặc biệt, nhóm xin gửi lời cảm ơn sâu sắc đến thầy PGS.TS. Nguyễn Thanh Hải. Thầy đã tận tình hướng dẫn, đồng hành cùng nhóm trong suốt quá trình nghiên cứu. Nhóm vô cùng trân trọng những ý kiến đóng góp quý báu, sự chỉ dẫn tận tâm cũng như tài liệu tham khảo hữu ích mà thầy đã cung cấp, giúp nhóm hoàn thiện bài báo cáo một cách tốt nhất.

Dù đã nỗ lực hết mình, nhưng bài báo cáo vẫn không tránh khỏi những thiếu sót. Nhóm rất mong nhận được sự thông cảm, góp ý từ thầy để tiếp tục hoàn thiện hơn trong tương lai.
Cuối cùng, nhóm xin kính chúc thầy cô luôn dồi dào sức khỏe, thành công trong sự nghiệp giảng dạy và nghiên cứu.

Trân trọng!
\chapter*{Lời cam đoan}
Nhóm chúng em xin cam đoan báo cáo là kết quả nghiên cứu của nhóm trong suốt thời gian qua. Tất cả số liệu, kết quả phân tích trong đề tài đều do nhóm tự tìm hiểu, nghiên cứu một cách khách quan, trung thực, có nguồn gốc rõ ràng và chưa từng được công bố dưới bất kỳ hình thức nào.

Nhóm chúng em cam kết nghiên cứu này không sao chép từ bất kỳ công trình nào khác. Mọi tài liệu tham khảo đều được trích dẫn đầy đủ theo quy định. Nhóm xin chịu hoàn toàn trách nhiệm nếu có bất kỳ sai sót hay sự không trung thực nào trong quá trình thực hiện đề tài.
Nghiên cứu của nhóm là trung thực, không sao chép từ các nghiên cứu khác. Có trích dẫn đầy đủ các tài liệu có tham khảo....

\chapter*{Danh mục bảng viết tắt}
\begin{table}[ht!]
\centering
\caption{Danh mục các ký hiệu viết tắt}\label{tab:data}
\begin{tabular}{|c|p{5cm}|p{7cm}|}
\hline
\large \textbf{Chữ viết tắt} & \large \textbf{Chữ đầy đủ }&  \large \textbf{Diễn giải} \\ \hline
ANN & Artificial Neural Network & Mạng nơ-ron nhân tạo \\
BoW & Bag of Words & Mô hình túi từ \\
CART & Classification and Regression Trees & Cây quyết định phân loại và hồi quy \\
CNN & Convolutional Neural Network & Mạng nơ-ron tích chập \\
CV & Coefficient of Variation & Hệ số biến thiên \\
EM & Expectation-Maximization & Thuật toán EM trong GMM \\
FNN & Feedforward Neural Network & Mạng truyền thẳng \\
GB & Gradient Boosting & Tăng cường độ dốc \\
GBM & Gradient Boosting Machine & Máy học tăng cường độ dốc \\
GDP & Gross Domestic Product & Tổng sản phẩm quốc nội \\
GMM & Gaussian Mixture Model & Mô hình hỗn hợp Gauss \\
HPG & Hoa Phat Group & Mã cổ phiếu của Hòa Phát Group \\
IQR & Interquartile Range & Khoảng tứ phân vị \\
KDE & Kernel Density Estimate & Ước lượng mật độ nhân \\
KNN & K-Nearest Neighbors & Láng giềng gần nhất \\
LSTM & Long Short-Term Memory & Mạng ghi nhớ dài-ngắn hạn \\
MAE & Mean Absolute Error & Sai số tuyệt đối trung bình \\
MAL & My Anime List & CSDL hoạt hình Nhật Bản \\
MAPE & Mean Absolute Percentage Error & Sai số phần trăm tuyệt đối trung bình \\
MSE & Mean Squared Error & Sai số bình phương trung bình \\
OOB & Out-of-Bag & Dữ liệu ngoài túi (Random Forest) \\
PCA & Principal Component Analysis & Phân tích thành phần chính \\
RF & Random Forest & Rừng ngẫu nhiên \\
RMSE & Root Mean Squared Error & Căn bậc hai của sai số bình phương trung bình \\
RNN & Recurrent Neural Network & Mạng nơ-ron hồi quy \\
RSNA & Radiological Society of North America & Hiệp hội X-quang Bắc Mỹ \\
SD & Standard Deviation & Độ lệch chuẩn \\
SGD & Stochastic Gradient Descent & Gradient descent ngẫu nhiên \\
SIIM & Society for Imaging Informatics in Medicine & Hội Tin học Hình ảnh trong Y học \\
STM & Short-Term Memory & Trí nhớ ngắn hạn \\
SVM & Support Vector Machine & Máy vector hỗ trợ \\
XGBoost & Extreme Gradient Boosting & Tăng cường độ dốc cực đại \\
\hline
\end{tabular}
\end{table}



\chapter*{Tóm tắt}
Môn học Phân tích Dữ liệu cung cấp cái nhìn toàn diện về các phương pháp phân tích dựa trên dữ liệu với nhiều loại dữ liệu khác nhau như dữ liệu dạng bảng, chuỗi thời gian và dữ liệu hình ảnh. Thông qua các ứng dụng thực tiễn và tập dữ liệu thực tế, sinh viên được học cách áp dụng các kỹ thuật thống kê và thuật toán học máy để khai thác thông tin và đưa ra quyết định dựa trên dữ liệu. Với dữ liệu dạng bảng, các phương pháp phân loại và hồi quy được sử dụng để phân tích kết quả học tập, đặc điểm tính cách và hành vi người dùng. Dữ liệu chuỗi thời gian như khí hậu, lịch sử khám chữa bệnh và giá cổ phiếu được phân tích bằng các mô hình dự báo và phát hiện bất thường. Dữ liệu hình ảnh, bao gồm ảnh chụp y tế và dữ liệu phát hiện cháy, được xử lý qua mạng nơ-ron tích chập (CNN) để thực hiện các tác vụ phân loại và phân đoạn. Môn học nhấn mạnh vào tiền xử lý dữ liệu, trực quan hóa, đánh giá mô hình và diễn giải kết quả, giúp sinh viên trang bị kỹ năng phân tích dữ liệu thực tế.


\textbf{Từ khóa}: \textit{Phân tích dữ liệu, thống kê, máy học,}

